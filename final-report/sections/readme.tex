\subsection{Project Description}

The purpose of this system is to increase security and safety of secured
locations, like offices. The system will attempt to prevent the spread of
illness by limiting the number of people which can enter the building, and by
preventing people with a high body temperature from entering the building.

This system will be mounted on the entrances and exits of a building. The system
will be modular to allow an arbitrary number of nodes to be added.

The system will display a green light at the entrance if it is accepting people.
The light will turn red when the area is closed or if there are too many people
inside.

If the light is green, the user will present their NFC security card to the
system. The system send a message containing a unique user ID to a server. The
server will use it’d database to determine if the user is authorized. The server
will then notify the door with the results of the query. If the user has not
already been denied, the system will then take the user’s temperature and send
the result to the server. If the temperature is below the set threshold, and
they have access, the server will allow the door to open.

Every access attempt, including the associated temperature reading and user ID,
is sent to the server to be logged in a database.

Users will also have to present their NFC card to the door to leave the
building. By tracking users entering and exiting the building, the server will
know how many people are in the building.

The central server also provides a GUI available to the building’s security
team.

\subsection{Repository Roadmap}

This repository contains all of the files related to our project. Our proposal,
design document and final report are located in the proposal, design and
final-report folders respectively. All of our source code is located in the src
folder.

Within the src folder, several top level programs are located such as
door\_node\_controller.py and control\_server.py. Aditional scripts are located
in subfolders of the src folder such as hardware drivers in the hardware folder,
GUI related code in the gui folder, ThingSpeak and messages parsing related
code in the communication folder and hardware stubs in the stub folder.

The test subfolder of the src directory of our repository contains all of our
test code for our end to end and testing demos.

\subsection{How to run the code}

\subsubsection{Dependencies}

The following dependancies are required to run the door node software with
hardware emulating stubs, the control server and the GUI server.

\begin{itemize}
    \item python $\geq$ 3.7
    \item python-requests
    \item python-flask
    \item python-flask-socketio
    \item python-gevent
    \item python-socketio
\end{itemize}

\noindent
In order to run the door node software with hardware present, the following
additional Python packages are required:

\begin{itemize}
    \item rpi.gpio
    \item mfrc522
    \item adafruit-circuitpython-vl53l0x
    \item smbus
\end{itemize}

\subsubsection{Control Server}

You will need to run two separate processes.

\noindent
The first is the Control Server which is responsible for the control logic.

\begin{verbatim}
cd src
./control_server.py
\end{verbatim}

\noindent
The second is the GUI which can be used to view and modify the settings of the
Control Server. It can also be used to view access logs.

\begin{verbatim}
cd src
./gui/run_gui.py
\end{verbatim}

\subsubsection{Door Node}

\begin{verbatim}
cd src
./door_node_controller.py
\end{verbatim}

