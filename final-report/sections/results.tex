\subsection{Changes from Proposal}
There are a variety of changes made from the initial project proposal. The initial project features that involved sending SMS and email notifications were not pursued. A decision was also made to change the functionality of the LED to account for adding an indication of whether or not the user must take their temperature. Due to time constraints and the removal of external messaging of users regarding the building the quarantine for 14 days system was removed.

The two changes for the door node were the addition of a distance sensor to have hardware used by each group member and changes to the LED. The change for the LED was the decision to use the off and yellow through combination of red and green to indicate message being transmitted or full capacity, and to indicate that information was being taken respectively. The distance sensor is meant to make sure that the temperature sensor is not touching the user and is close enough to take an accurate measurement. The Final door actuator unlike the proposal is not set up with a fail-safe. With further work on the project it was decided to be unnecessary to have multiple levels of access that the security system would check for as the temperature check is unnecessary for monitoring at places within the building that require additional security clearance.

A change in the final version of the GUI was that our group chose to rather then have the ability to override the door actuator and approve quarantining to instead be able to add who is a valid user and set new access points for if a new door node is installed. Due to time constraints it was decided to not pursue creating a 3D printed door as a model to showcase. There was a decision that both for time and ease of communication to change the hardware interface stubs from creating log files to instead print out statements of their actions and send those outputs to the database as though they were data taken from original hardware. The choice was made for practical reasons to not have a test rig that allowed for the temperature sensor to test controlled arbitrary temperatures. Instead we opted to in testing have limits that were far apart to deal with the fact that the temperature sensor gave lower temperatures then expected. Connected with the temperature sensor the temperature sensor chosen did not require a calibration set up.

The schedule for the project was heavily modified after the submission of Project Design. The time restrictions lead to the removal of the Door Model being created and the milestones being changed to be less concrete and less formal deadlines that were focused on having the necessary code ready for the three demonstrations becoming the priority.
