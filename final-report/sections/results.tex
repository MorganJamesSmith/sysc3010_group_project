\subsection{Changes from Proposal}
\subsubsection{Project Description}
There are a variety of changes made from the initial project proposal. The initial project features that involved sending SMS and email notifications were not pursued. A decision was also made to change the functionality of the LED to account for adding an indication of whether or not the user must take their temperature. Due to time constraints and the removal of external messaging of users regarding the building the quarantine for 14 days system was removed.
\subsubsection{Proposed Solution}
The two changes for the door node were the addition of a distance sensor to have hardware used by each group member and changes to the LED. The change for the LED was the decision to use the off and yellow through combination of red and green to indicate message being transmitted or full capacity, and to indicate that information was being taken respectively. The distance sensor is meant to make sure that the temperature sensor is not touching the user and is close enough to take an accurate measurement. The Final door actuator unlike the proposal is not set up with a fail-safe. With further work on the project it was decided to be unnecessary to have multiple levels of access that the security system would check for as the temperature check is unnecessary for monitoring at places within the building that require additional security clearance. A change in the final version of the GUI was that our group chose to rather then have the ability to override the door actuator and approve quarantining to instead be able to add who is a valid user and set new access points for if a new door node is installed.
\subsubsection{Team Roles and Testing}
 Due to time constraints it was decided to not pursue creating a 3D printed door as a model to showcase. There was a decision that both for time and ease of communication to change the hardware interface stubs from creating log files to instead print out statements of their actions and send those outputs to the database as though they were data taken from original hardware. The choice was made for practical reasons to not have a test rig that allowed for the temperature sensor to test controlled arbitrary temperatures. Instead we opted to in testing have limits that were far apart to deal with the fact that the temperature sensor gave lower temperatures then expected. Connected with the temperature sensor the temperature sensor chosen did not require a calibration set up.
\subsubsection{Schedule}
The schedule for the project was heavily modified after the submission of Project Design. The time restrictions lead to the removal of the Door Model being created and the milestones being changed to be less concrete and less formal deadlines that were focused on having the necessary code ready for the three demonstrations becoming the priority.

\subsection{Findings}


\subsubsection{ThingSpeak}

One of the most difficult to debug aspects of our project was the ThingSpeak
communication code. In our project we used ThingSpeak for communication between
nodes. Because the nature of our project did not match well with the structure
of ThingSpeak we wrote an abstraction layer over our ThingSpeak channel to
use it as a medium over which we could provide a communication service similar
to a TCP or UDP socket. This abstraction made our application much easier to
write, but we also ran into some quarks with ThingSpeak that complicated the
abstraction itself.

The REST API documentation on the ThingSpeak site is not very complete. While it
describes the operation of the API in broad stokes, there are many minor details
missing. One of those is the that ThingSpeak has a rate limit for posts to
channels. This lack of documentation is compounded by the fact that ThingSpeak
does not return an error when an attempt is made to update a channel too
quickly, it simply drops the request silently. It took us a while to discover
that intermittent problems with missing channel entries where caused by a rate
limit end then to characterize the rate limit as being one update per second.
This discovery led us to implement a mechanism in our ThingSpeak code such that
when sending a packet out code verifies that it has appeared on the ThingSpeak
channel, and if it is not appeared after a presepcified delay the entry will
be resubmitted.

\subsubsection{Sensors}

We used two sensors in our project, an infrared temperature sensor and an
infrared range finder. We performed testing with both of these sensors in an
attempt to characterize them and get an idea of what values we could expect to
read from them in normal operation for our use case.

We found that our infrared temperature sensor consistently returned measurements
much lower than what we expected. Normal human body temperature is around 37
degrees Celsius, and we had assumed that we would see readings that where a few
degrees below this because we where measuring skin temperature rather than
body temperature. In practice, when measuring forehead temperature we saw values
ranging from approximately 26 to 31 degrees. This appears to be the nature of
forehead temperature measurements rather than an issue with our particular sensor
since the issue persisted after swapping to a different temperature sensor of the
same model and we got expected measurements for object other than people. We
where unable to source the medical grade version of the sensor we used, and had
we been able to would likely have seen slightly smaller variability if we had
used the medical rather than industrial calibrated version, but that does not
account for all of the very large variability we saw in practice.


The lower temperatures and much larger variability than expected left us unable
to determine a reasonable threshold for fever detection, if we wanted to work
further on this project we would have to invest more resources into taking a
wide variety of measurements in order to get a better idea of where temperature
thresholds should be set. If we had access to a proper black body radio source
we would have been able to perform better calibration of our sensor as well.

In contrast to the temperature sensor, the infrared range finder worked as
expected. The sensor driver provided us a value in millimetres which in our
tests was typically within about ten millimetres away from the true distance we
where testing with. This accuracy was more than sufficient for our use case.


