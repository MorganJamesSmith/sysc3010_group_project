\subsection{End to End Demo}

\noindent
Hardware requirements:
\begin{itemize}
    \item Setup Raspberry Pi's in headless mode 
    \item Connect sensors to Raspberry Pi's
    \item If available, connect LED displays to Raspberry Pi
\end{itemize}

\noindent
Software requirements:
\begin{itemize}
    \item Set up Thingspeak channel 
    \item Hard code sensor signal communication to Thingspeak 
    \item Have test stubs which emulate the functionality of the hardware
          components ready
\end{itemize}

\subsubsection*{Entry Access}

\paragraph{Scenario 1}
Employee with valid NFC card, normal body temperature and ideal distance away
from temperature sensor at entry node.

\noindent
Features to be tested:
\begin{itemize}
    \item Communication between the NFC card reader and Thingspeak 
    \item Communication between control server and Thingspeak
    \item Communication between Time of Flight range finder and Thingspeak 
    \item Communication between Time of Flight range finder and Thingspeak 
    \item Communication between LED display and Thingspeak 
    \item Communication between Temperature sensor and Thingspeak 
    \item Communication between control server and database
    \item Communication between Thingspeak and the electronic lock at entry node
\end{itemize}

\noindent
Test scenario steps:
\begin{enumerate}
    \item Control server uses Thingspeak to send the LED display a signal to
          display green light to indicate door node accepting entries
    \item NFC reader identifies NFC id card being tapped and notifies control
          server of access attempt
    \item NFC reader identifies tapped NFC card id and reports the id to the
          control server
    \item Control server recognizes employee id based on NFC card id
    \item Control server signals the rangefinder to identify if employee in
          range of temperature sensor through Thingspeak
    \item Range finder reports back to the control server with the results and
          displays orange light
    \item When in range, the control server sends a signal to the temperature
          sensor to measure employee body temperature through Thingspeak
    \item Temperature sensor sends the recorded temperature to the control
          server using Thingspeak
    \item Control server saves the recorded temperatures and other required
          information into the database
    \item Control server compares recorded temperature to acceptable temperature
          range
    \item If in acceptable temperature range, control server authorizes employee
          to enter building through Thingspeak
    \item Control server sends a signal to the led display to display a flashing
          green light to indicate employee's access has been authorized
    \item Control server signals the electronic lock to unlock and adds 1 to
          number of people on the premises through Thingspeak
    \item Control server uploads the additional entry attempt recorded data to
          database
\end{enumerate}

\noindent
Expected test scenario result: proper communication between nodes and control
server occurred to authorize employee access to the workplace.


\paragraph{Scenario 2}
Employee with valid NFC card, normal body temperature and not in ideal distance
range away from temperature sensor at entry node.

\noindent
Features to be tested:
\begin{itemize}
    \item Communication between the NFC card reader and Thingspeak
    \item Communication between control server and Thingspeak
    \item Communication between Time of Flight range finder and Thingspeak 
    \item Communication between LED display and Thingspeak 
    \item Communication between Thingspeak and the electronic lock at entry node
\end{itemize}

\noindent
Test scenario steps:
\begin{enumerate}
    \item Control server uses Thingspeak to send the LED display a signal to
          display green light to indicate door node accepting entries
    \item NFC reader identifies NFC id card being tapped and notifies control
          server of access attempt
    \item NFC reader identifies tapped NFC card id and reports the id to the
          control server
    \item Control server recognizes employee id based on NFC card id
    \item Control server signals the range-finder to identify if employee in
          range of temperature sensor through Thingspeak
    \item Range finder reports back to the control server with the results and
          displays red light to let employee know to stand in ideal range away
          from the temperature sensor
    \item Control server signals electronic lock to stay locked through
          Thingspeak
\end{enumerate}

\noindent
Expected test scenario result: proper communication between nodes and control
server occurred to let the employee know to stand in an ideal distance range
away from the temperature sensor.

\paragraph{Scenario 3}
Employee with valid NFC card, body temperature over 39.1 °C and in ideal
distance range away from temperature sensor at entry node.

\noindent
Features to be tested:
\begin{itemize}
    \item Communication between the NFC card reader and Thingspeak 
    \item Communication between control server and Thingspeak
    \item Communication between Time of Flight range finder and Thingspeak 
    \item Communication between LED display and Thingspeak 
    \item Communication between Temperature sensor and Thingspeak 
    \item  Communication between the GUI and Thingspeak
    \item Communication between control server and database
    \item Communication between Thingspeak and the electronic lock at entry node
\end{itemize}

\noindent
Test scenario steps:
\begin{enumerate}
    \item Control server uses Thingspeak to send the LED display a signal to
          display green light to indicate door node accepting entries
    \item Identify tapped NFC card id
    \item Notify control server of access attempt
    \item NFC reader recognizes employee id based on NFC card id and reports to
          control server through Thingspeak
    \item Control server signals the range-finder to identify if employee in
          range of temperature sensor through Thingspeak
    \item Range finder reports back to the control server with the results and
          displays orange light if employee in range of the temperature sensor
          and red light if employee not in range 
    \item When in range, the control server sends a signal to the temperature
          sensor to measure employee body temperature through Thingspeak
    \item Temperature sensor sends the recorded temperature to the control
          server using Thingspeak
    \item Control server saves the recorded temperatures and other required
          information into the database
    \item Control server compares recorded temperature to acceptable temperature
          range
    \item If not in ideal temperature range, control server restricts employee
          from entering the building through Thingspeak
    \item Control server sends a signal to the led display to display a flashing
          red light to indicate employee's access has been unauthorized
    \item Control server sends a quarantine approval notification to the GUI
          through Thingspeak
    \item User approves the quarantine period notification
    \item Approval is sent to the control server through Thingspeak from the GUI
    \item Control server adds the employee information and recorded data to
          unauthorized employee table in the database and marks entry access
          date as the start of quarantine date 
    \item Using Thingspeak the control server signals the electronic lock to
          stay locked 
    \item Control server uploads the additional entry attempt recorded data to
          database
\end{enumerate}

\noindent
Expected test scenario result: proper communication between nodes and control
server occurred to unauthorize employee access to workplace.

\subsubsection*{Exit Access}

\paragraph{Scenario 1}
Employee with valid NFC card and at exit node.

\noindent
Features to be tested:
\begin{itemize}
    \item Communication between the NFC card reader and Thingspeak
    \item Communication between the control server and Thingspeak
    \item Communication between Thingspeak and the electronic lock at exit node 
\end{itemize}

\noindent
Test scenario steps:
\begin{enumerate}
    \item NFC reader identifies NFC id card being tapped and notifies control
          server of exit attempt
    \item NFC reader identifies tapped NFC card id and reports the id to the
          control server
    \item Control server recognizes employee id based on NFC card id 
    \item Control server logs the data recorded for the exit attempt and updates
          the database 
    \item Control server signals the electronic lock to unlock and subtracts 1
          from the number of people on the premises through Thingspeak
\end{enumerate}

\noindent
Expected test scenario result: employee successfully exits the building.

\subsection{Hardware Testing Demo}

\subsubsection{Infrared Temperature Sensor}

\begin{table*}[htb]
\centering
\begin{tabular}{>{\centering\arraybackslash}m{0.75cm}|>{\centering\arraybackslash}m{4cm}|>{\centering\arraybackslash}m{4.5cm}|>{\centering\arraybackslash}m{4cm}}
\toprule
Test & Description & Setup & Expected Result \\
\midrule
1 & Measure a temperature within the normal range & Aim temperature sensor at
someone's forehead and take a measurement & Measured temperature value is within
the expected range \\
\hline
2 & Measure a temperature that could indicate a fever & Aim the temperature
sensor at a pot of water that has been heated to 39 ℃ and take a measurement &
Measured temperature value should be with 0.5 ℃ of 39 ℃. \\
\hline
3 & Measure a temperature below the expected range & Aim the temperature sensor
at any room temperature surface and take a measurement & The measurement should be
approximately room temperature. \\
\bottomrule
\end{tabular}
\caption{Infrared Temperature Sensor Tests}
\label{table:ir-tests}
\end{table*}

\subsubsection{NFC Security Badge Reader}

\begin{table*}[htb]
\centering
\begin{tabular}{>{\centering\arraybackslash}m{0.75cm}|>{\centering\arraybackslash}m{4cm}|>{\centering\arraybackslash}m{4.5cm}|>{\centering\arraybackslash}m{4cm}}
\toprule
Test & Description & Setup & Expected Result \\
\midrule
1 & Green RGB LED functions & Have breadboard set up & Green is shown on LED \\
\hline
2 & Red RGB LED functions & Have breadboard set up & Red is shown on LED \\
\hline
3 & Green RGB LED can change to red & Have breadboard set up & Initially green
is shown on LED which then changes to red \\
\hline
4 & Red RGB LED can change to green & Have breadboard set up & Initially red is
shown on LED which then changes to green \\
\hline
5 & MFRC 522 RFID Module activates & Have breadboard set up. Have a passive tag
ready to wave in front & Pi prints out data on RFID tag \\
\hline
6 & MFRC 522 RFID Module can note separate ID & Have breadboard set up. Have two
passive tags ready to wave in front & Pi prints out different data for each RFID
tag swiped in front. \\
\hline
7 & MFRC 5222 RFID Module will not read when RGB LED is red & Have breadboard
set up. Have a passive tag ready to wave in front & RGB LED is red and no data
is printed out by the Pi. \\
\bottomrule
\end{tabular}
\caption{NFC Security Badge Reader Sensor Tests}
\label{table:nfc-tests}
\end{table*}

\subsubsection{Electronic Door Lock}

\begin{table*}[htb]
\centering
\begin{tabular}{>{\centering\arraybackslash}m{0.75cm}|>{\centering\arraybackslash}m{4cm}|>{\centering\arraybackslash}m{4.5cm}|>{\centering\arraybackslash}m{4cm}}
\toprule
Test & Description & Setup & Expected Result \\
\midrule
1 & Test servo Oscillation & Have breadboard set up & Servo alternates between
0 degrees and 180 degrees every second \\
\bottomrule
\end{tabular}
\caption{Electronic Door Lock Tests}
\label{table:servo-tests}
\end{table*}


A test program will be written that will initially place the servo at an angle
of 0 degrees.  Then a loop will run that will move the servo arm to 180 degrees,
wait a second, move back to 0 degrees, and wait another second.  The operation
of the servo can be easily verified visually.

\subsubsection{Time of Flight Range Sensor}

\begin{table*}[htb]
\centering
\begin{tabular}{>{\centering\arraybackslash}m{0.75cm}|>{\centering\arraybackslash}m{4cm}|>{\centering\arraybackslash}m{4.5cm}|>{\centering\arraybackslash}m{4cm}}
\toprule
Test & Description & Setup & Expected Result \\
\midrule
1 & Orange\_LED function activated & Have LED display set up with the Raspberry
Pi connected to monitor & RGB LED displays Orange \\
\hline
2 & Red\_LED function activated & Have LED display set up with the Raspberry Pi
connected to monitor & RGB LED displays Red \\
\hline
3 & Switch between orange and red LEDs & Have LED display set up with the
Raspberry Pi connected to monitor & RGB LED initially displays orange and then
switches to display red \\
\hline
4 & Switch between red and orange LEDs & Have LED display set up with the
Raspberry Pi connected to monitor & RGB LED initially displays red and then
switches to display orange \\
\hline
5 & VL6180X module activated & Have sensor connected to the Raspberry Pi and
connect Raspberry Pi to a display & Display prints out initial reading \\
\hline
6 & VL6180X module activated and Range\_Find class used to display multiple
range recordings & Have sensor connected to the Raspberry Pi and connect
Raspberry Pi to a display. Keep moving hand over the sensor to be able to
measure range. & Display prints out different readings recorded from moving
hand over the sensor \\
\bottomrule
\end{tabular}
\caption{Time of Flight Sensor Tests}
\label{table:tof-tests}
\end{table*}

\subsection{Database Demo}

The following features are to be tested in order to test the integrity of the
database:

\begin{itemize}
    \item Control server populating right tables 
    \item Proper formatted data uploaded to appropriate fields
    \item Records of the temperature readings of all employees are accessible
          for at least 14 days
\end{itemize}

\paragraph{Scenario 1}
Populating sample test data access\_entry table.

\noindent
Features to be tested:
\begin{itemize}
    \item Control server populating right tables
\end{itemize}

\noindent
Initial setup: set up Raspberry pi connected to monitor keyboard and mouse.
Open command prompt on Raspberry Pi.

\noindent
Test scenario steps:
\begin{enumerate}
    \item Invoke SQLite by typing this command in the command prompt
\begin{lstlisting}
sqlite3
\end{lstlisting}
    \item Open the database by typing this command 
\begin{listing}[H]
\begin{minted}[]{text}
.open project_database.db
\end{minted}
\end{listing}
    \item Populate data to employee access\_entry table
\begin{listing}[H]
\begin{minted}[]{sql}
INSERT INTO access_entry
    VALUES(19234,time('now'),date('now'), "S_entry",
           "Yes", 38.0, "Authorized");
INSERT INTO access_entry
    VALUES(19437,time('now','-12 minutes'),date('now'),
          "E_entry", "Yes", 38.3, "Authorized");
INSERT INTO access_entry
    VALUES(19632,time('now','-1 hour'),date('now'),
           "N_entry", "Yes", 39.2, "Unauthorized");
INSERT INTO access_entry
    VALUES(19957,time('now','-5 minutes'),date('now'),
           "W_entry", "Yes", 38.5, "Authorized");
\end{minted}
\end{listing}
    \item Query data using \lstinline{Select*} command
\begin{listing}[H]
\begin{minted}[]{sql}
SELECT * FROM access_entry;
\end{minted}
\end{listing}
\end{enumerate}

\noindent
Expected test scenario result: table populated with test sample data and query
outputs.

\begin{lstlisting}
19234 | 09:30:23 | 2020-10-26 | S_entry | Yes | 38.0 | Authorized
19437 | 09:18:23 | 2020-10-26 | E_entry | Yes | 38.3 | Authorized
19632 | 08:30:23 | 2020-10-26 | N_entry | Yes | 39.2 | Unauthorized
19957 | 09:25:23 | 2020-10-26 | W_entry | Yes | 38.5 | Authorized
\end{lstlisting}

\paragraph{Scenario 2}
Populating sample test data nfc\_and\_employer\_id table.

\noindent
Features to be tested:
\begin{itemize}
    \item Proper formatted data uploaded to appropriate fields
\end{itemize}

\noindent
Initial setup: set up Raspberry pi connected to monitor keyboard and mouse.
Open command prompt on Raspberry Pi.

\noindent
Test scenario steps:
\begin{enumerate}
    \item Invoke SQLite by typing this command in the command prompt
\begin{listing}[H]
\begin{minted}[]{bash}
sqlite3
\end{minted}
\end{listing}
    \item Open the database by typing this command
\begin{listing}[H]
\begin{minted}[]{sql}
.open project_database.db
\end{minted}
\end{listing}
    \item Populate data to employee access\_entry table
\begin{listing}[H]
\begin{minted}[]{sql}
INSERT INTO access_entry
    VALUES(19234,time('now'),date('now'), "S_entry",
           "Yes", 38.0, "Authorized");
INSERT INTO access_entry
    VALUES(19437,time('now','-12 minutes'),date('now'),
           "E_entry","Yes", 38.3, "Authorized");
INSERT INTO access_entry
    VALUES(19632,time('now','-1 hour'),date('now'),
           "N_entry","Yes", 39.2, "Unauthorized");
INSERT INTO access_entry
    VALUES(19957,time('now','-5 minutes'),date('now'),
           "W_entry","Yes", 38.5, "Authorized");
\end{minted}
\end{listing}
    \item Query data using \lstinline{Select*} command
\begin{listing}[H]
\begin{minted}[]{sql}
SELECT access_time, access_date FROM access_entry;
\end{minted}
\end{listing}
\end{enumerate}

\noindent
Expected test scenario result: table populated with test sample data and query
outputs.

\begin{listing}[H]
\begin{minted}[]{text}
09:30:23 | 2020-10-26
09:18:23 | 2020-10-26
08:30:23 | 2020-10-26
09:25:23 | 2020-10-26
\end{minted}
\end{listing}

\paragraph{Scenario 3}
Populating sample test data access\_entry table.

\noindent
Features to be tested:
\begin{itemize}
    \item Records of the temperature readings of all employees are accessible
          for at least 14 days
\end{itemize}

\noindent
Initial setup: set up Raspberry pi connected to monitor keyboard and mouse.
Open command prompt on Raspberry Pi.

\noindent
Test scenario steps:
\begin{enumerate}
    \item Invoke SQLite by typing this command in the command prompt
\begin{listing}[H]
\begin{minted}[]{bash}
sqlite3
\end{minted}
\end{listing}
    \item Open the database by typing this command
\begin{listing}[H]
\begin{minted}[]{sql}
.open project_database.db
\end{minted}
\end{listing}
    \item Populate data to employee access\_entry table
\begin{listing}[H]
\begin{minted}[]{sql}
INSERT INTO access_entry
    VALUES (19234, time('now'), date('now'), "S_entry",
            "Yes", 38.0, "Authorized");
INSERT INTO access_entry
    VALUES (19437, time('now','-12 minutes'),
            date('now', '-1 day'), "E_entry", "Yes", 38.3,
            "Authorized");
INSERT INTO access_entry
    VALUES (19632, time('now','-1 hour'),
            date('now', '-3 days'), "N_entry", "Yes", 39.2,
            "Unauthorized");
INSERT INTO access_entry
    VALUES (19957, time('now','-5 minutes'),
            date('now', '-7 days'), "W_entry", "Yes", 38.5,
            "Authorized");
INSERT INTO access_entry
    VALUES (19234, time('now'), date('now', '-13 days'),
            "S_entry", "Yes", 38.0, "Authorized");
INSERT INTO access_entry
    VALUES (19437, time('now','-12 minutes'),
            date('now', '-14 days'), "E_entry", "Yes", 38.3,
            "Authorized");
INSERT INTO access_entry
    VALUES (19632, time('now','-1 hour'),
            date('now', '-14 days'), "N_entry", "Yes", 39.2,
            "Unauthorized");
INSERT INTO access_entry
    VALUES (19957, time('now','-5 minutes'),
            date('now', '-15 days'), "W_entry", "Yes", 38.5,
            "Authorized");
\end{minted}
\end{listing}
    \item Query data using \lstinline{Select*} command
\begin{listing}[H]
\begin{minted}[]{sql}
SELECT * FROM access_entry WHERE access_date > date('now');
\end{minted}
\end{listing}
\end{enumerate}

\noindent
Expected test scenario result: table populated with test sample data and query
outputs.

\begin{listing}[H]
\begin{minted}[]{text}
19437 | 09:18:23 | 2020-10-25 | E_entry | Yes | 38.3 | Authorized
19632 | 08:30:23 | 2020-10-23 | N_entry | Yes | 39.2 | Unauthorized
19957 | 09:25:23 | 2020-10-19 | W_entry | Yes | 38.5 | Authorized
19234 | 09:30:23 | 2020-10-13 | S_entry | Yes | 38.0 | Authorized
19437 | 09:18:23 | 2020-10-12 | E_entry | Yes | 38.3 | Authorized
19632 | 08:30:23 | 2020-10-12 | N_entry | Yes | 39.2 | Unauthorized
19957 | 09:25:23 | 2020-10-11 | W_entry | Yes | 38.5 | Authorized
\end{minted}
\end{listing}


\subsection{Final Demo}

Table \ref{table:final-tests} lists the scenarios that we will test during the
final demo. These scenarios have been chosen to demonstrate the functional
requirements as listed in section \ref{sec:problem-statement}.

% Functional requirements:
% - Control access to a building using security badges.
% - Require users to present their security badge when they enter or exit the
%   building in order to track the number of users in the building.
% - Measure user's temperatures when they are entering the building in order to
%   determine if they have possible symptoms.
% - The door node should have a range sensor to determine whether users are in
%   an appropriate position for a temperature reading.
% - Do not allow more users to enter the building if the a preset maximum
%   capacity has been reached.
% - An multicoloured LED at each door node should indicate be used to indicate
%   the status of the door node. The LED should be normally red when the door is
%   locked and should change to green when the door is unlocked. The LED should
%   be orange when in the process of taking a temperature reading if the user is
%   not within an appropriate range of the temperature sensor.


\begin{longtable}[htb]{>{\centering\arraybackslash}m{3cm}|>{\centering\arraybackslash}m{3.5cm}|>{\centering\arraybackslash}m{3cm}|>{\centering\arraybackslash}m{3.5cm}}
\toprule
Description & Requirement & Procedure & Expected Result \\
\midrule
Ideal Building Entry & Control access to a building using security badges,
measure users' temperatures when they are entering the building, change LED 
colours & Present a valid security badge to the incoming reader, provide a
normal temperature reading the temperature sensor & The electronic door lock
should be actuated to allow entry, the access should be logged in the database,
the count of people in the building should incremented and the LED at the door
node should change to green temporarily \\
\hline
Ideal Building Exit & Require users to present their security badge when exiting
the building in order to track the number of users in the building & Present a
valid building security badge to the outgoing reader at a door node & The door
node should be actuated to allow the user to exit, the exit should be logged in
the database and the count of people in the building should be decremented \\
\hline
Attempted building entry with invalid security badge & Control access to a
building using security badges & Present an invalid security badge to the
incoming reader & The electronic door lock should not be actuated and entry
should not be allowed, the attempted access should be logged in the database and
the LED at the door node should remain red \\
\hline
Attempted building entry with fever & Measure user's temperature when they are
entering the building & Present a valid security badge to the reader but then
present a temperature that is too high & The electronic door lock should not be
actuated and entry should not be allowed, the attempted access should be logged
in the database and the LED at the door node should remain red \\
\hline
Attempted building entry with low temperature & Measure user's temperature when
they are entering the building & Present a valid security badge to the reader
but then present a temperature that is too low & The electronic door lock should
not be actuated and entry should not be allowed, the attempted access should be
logged in the database and the LED at the door node should remain red \\
\hline
Attempted building entry when building is at capacity & Do not allow users to
enter the building if a maximum capacity has been reached & Set the maximum
capacity of the system to one users, enter with a valid security badge and
temperature reading then attempt to enter with a second valid security badge &
The electronic door lock should not be actuated and entry should not be allowed,
the attempted access should be logged in the database and the LED at the door
node should remain red \\
\hline
Temperature measurement from incorrect distance & Determine whether users are in
an appropriate position for a temperature reading & Present a valid security
badge to the incoming reader but initially stand approximately one meter away
from the temperature sensor, slowly move closer until a temperature reading is
taken & The LED should be illuminated orange and the electronic door lock should
not be actuated until the user is within an appropriate range of the infrared
temperature sensor \\
\bottomrule
\caption{Final Demo Tests}
\label{table:final-tests}
\end{longtable}


