\subsection{End to End Demo}

\noindent
Software requirements:
\begin{itemize}
    \item Set up Thingspeak channel 
    \item Hard code sensor signal communication to Thingspeak 
\end{itemize}

\subsubsection*{Entry Access}

\paragraph{Scenario 1}
Building accepting entries, an employee with valid NFC card, authorized status,
normal body temperature, an ideal distance away from the temperature sensor and
building accessible after entry

\noindent
Features to be tested:
\begin{itemize}
    \item Communication between control server and Thingspeak
    \item Communication between door node and Thingspeak
\end{itemize}

\noindent
Test scenario steps:
\begin{enumerate}
    \item Control server sends a DOOR\_STATE\_UPDATE message request with a new
          state to the Thingspeak channel
    \item Door node receives the DOOR\_STATE\_UPDATE from Thingspeak channel 
    \item Door node sends ACCESS\_REQUEST with transaction id and NFC security
          badge id to Thingspeak channel
    \item Control server receives ACCESS\_REQUEST from Thingspeak channel 
    \item Control server sends an INFROMATION\_REQUEST with transaction ID and
          information type to the Thingspeak channel
    \item Door node receives INFROMATION\_REQUEST from the Thingspeak channel 
    \item Door node sends an INFORMATION\_RESPONSE with the transaction ID,
          information type and information value to the Thingspeak channel
    \item Control server receives the INFORMATION\_RESPONSE from the Thingspeak
          channel
    \item Control server sends an ACCESS\_RESPONSE with transaction id and
          access response type to the Thingspeak channel
    \item Door node receives the ACCESS\_REPONSE from the Thingspeak channel
    \item Control server sends a DOOR\_STATE\_UPDATE message request with a new
          state to the Thingspeak channel
    \item Door node receives the DOOR\_STATE\_UPDATE from Thingspeak channel 
\end{enumerate}

\noindent
Expected test scenario result: proper communication between nodes and control
server has occurred.

\paragraph{Scenario 2}
An employee with a valid NFC card and at the exit node.

\noindent
Features to be tested:
\begin{itemize}
    \item Communication between control server and Thingspeak
    \item Communication between door node and Thingspeak
\end{itemize}

\noindent
Test scenario steps:
\begin{enumerate}
    \item Door node sends an ACCESS\_REQUEST with transaction id and NFC
          security badge id to Thingspeak channel
    \item Control server receives ACCESS\_REQUEST from Thingspeak channel 
    \item control server sends ACCESS\_RESPONSE with transaction id and access
          response type to the Thingspeak channel
    \item Door node receives the ACCESS\_REPONSE from the Thingspeak channel
    \item Control server sends a DOOR\_STATE\_UPDATE message request with a new
          state to the Thingspeak channel
    \item Door node receives the DOOR\_STATE\_UPDATE from Thingspeak channel 
\end{enumerate}

\noindent
Expected test scenario result: Proper communication between door node and
control server occurred

\subsection{Testing Demo}

\subsubsection{Hardware Tests}

\paragraph{Infrared Temperature Sensor}

Table \ref{table:ir-tests} shows the test cases that will be used to test the
infrared temperature sensor.

\begin{longtable}[htb]{>{\centering\arraybackslash}m{0.75cm}|>{\centering\arraybackslash}m{4cm}|>{\centering\arraybackslash}m{4.5cm}|>{\centering\arraybackslash}m{4cm}}
\toprule
Test & Description & Setup & Expected Result \\
\midrule
1 & Measure a temperature within the normal range & Aim temperature sensor at
someone's forehead and take a measurement & Measured temperature value is within
the expected range \\
\hline
2 & Measure a temperature that could indicate a fever & Aim the temperature
sensor at a pot of water that has been heated to 39 ℃ and take a measurement &
Measured temperature value should be with 0.5 ℃ of 39 ℃. \\
\hline
3 & Measure a temperature below the expected range & Aim the temperature sensor
at any room temperature surface and take a measurement & The measurement should
be approximately room temperature. \\
\bottomrule
\caption{Infrared Temperature Sensor Tests}
\label{table:ir-tests}
\end{longtable}

\paragraph{NFC Security Badge Reader}

Table \ref{table:nfc-tests} shows the test cases that will be used to test the
NFC security badge reader.

\begin{longtable}[htb]{>{\centering\arraybackslash}m{0.75cm}|>{\centering\arraybackslash}m{4cm}|>{\centering\arraybackslash}m{4.5cm}|>{\centering\arraybackslash}m{4cm}}
\toprule
Test & Description & Setup & Expected Result \\
\midrule
1 & Green RGB LED functions & Have breadboard set up & Green is shown on LED \\
\hline
2 & Red RGB LED functions & Have breadboard set up & Red is shown on LED \\
\hline
3 & Green RGB LED can change to red & Have breadboard set up & Initially green
is shown on LED which then changes to red \\
\hline
4 & Red RGB LED can change to green & Have breadboard set up & Initially red is
shown on LED which then changes to green \\
\hline
5 & Red RGB LED can change to green & Have breadboard set up & Initially red is
shown on LED which then changes to being turned off \\
\hline
6 & MFRC 522 RFID Module activates & Have breadboard set up. Have a passive tag
ready to wave in front & Pi prints out data on RFID tag \\
\hline
7 & MFRC 522 RFID Module can note separate ID & Have breadboard set up. Have two
passive tags ready to wave in front & Pi prints out different data for each RFID
tag swiped in front. \\

\bottomrule
\caption{NFC Security Badge Reader Sensor Tests}
\label{table:nfc-tests}
\end{longtable}

\paragraph{Electronic Door Lock}

Table \ref{table:servo-tests} shows the test cases that will be used to test the
electronic door lock.

\begin{longtable}[htb]{>{\centering\arraybackslash}m{0.75cm}|>{\centering\arraybackslash}m{4cm}|>{\centering\arraybackslash}m{4.5cm}|>{\centering\arraybackslash}m{4cm}}
\toprule
Test & Description & Setup & Expected Result \\
\midrule
1 & Test servo Oscillation & Have breadboard set up & Servo alternates between
0 degrees and 180 degrees every second \\
\bottomrule
\caption{Electronic Door Lock Tests}
\label{table:servo-tests}
\end{longtable}

A test program will be written that will initially place the servo at an angle
of 0 degrees.  Then a loop will run that will move the servo arm to 180 degrees,
wait a second, move back to 0 degrees, and wait another second.  The operation
of the servo can be easily verified visually.

\paragraph{Time of Flight Range Sensor}

Table \ref{table:tof-tests} shows the test cases that will be used to test the
time of flight range sensor.

\begin{longtable}[htb]{>{\centering\arraybackslash}m{0.75cm}|>{\centering\arraybackslash}m{4cm}|>{\centering\arraybackslash}m{4.5cm}|>{\centering\arraybackslash}m{4cm}}
\toprule
Test & Description & Setup & Expected Result \\
\midrule
1 & Orange\_LED function activated & Have LED display set up with the Raspberry
Pi connected to monitor & RGB LED displays Orange \\
\hline
2 & Red\_LED function activated & Have LED display set up with the Raspberry Pi
connected to monitor & RGB LED displays Red \\
\hline
3 & Switch between orange and red LEDs & Have LED display set up with the
Raspberry Pi connected to monitor & RGB LED initially displays orange and then
switches to display red \\
\hline
4 & Switch between red and orange LEDs & Have LED display set up with the
Raspberry Pi connected to monitor & RGB LED initially displays red and then
switches to display orange \\
\hline
5 & VL6180X module activated & Have sensor connected to the Raspberry Pi and
connect Raspberry Pi to a display & Display prints out initial reading \\
\hline
6 & VL6180X module activated and Range\_Find class used to display multiple
range recordings & Have sensor connected to the Raspberry Pi and connect
Raspberry Pi to a display. Keep moving hand over the sensor to be able to
measure range. & Display prints out different readings recorded from moving
hand over the sensor \\
\bottomrule
\caption{Time of Flight Sensor Tests}
\label{table:tof-tests}
\end{longtable}

\subsubsection{Software Tests}

\paragraph{Message Format}

The unit tests described in table \ref{table:msg-frmt-tests} will be replicated
for each of the message classes (AccessRequestMessage, AccessResponseMessage,
InformationRequestMessage, InformationResponseMessage and
DoorStateUpdateMessage).

\begin{longtable}[htb]{>{\centering\arraybackslash}m{0.75cm}|>{\centering\arraybackslash}m{4cm}|>{\centering\arraybackslash}m{4.5cm}|>{\centering\arraybackslash}m{4cm}}
\toprule
Test & Description & Setup & Expected Result \\
\midrule
1 & Test the constructor & Create a new message object with the constructor &
Fields should be set to the values passed into the contructor \\
\hline
2 & Test convertion to byte array & Call the to\_bytes() method on a message
object & The resulting encoded mesage should be in the correct format \\
\hline
3 & Test parsing of valid packet & Use the \_parse() method to create a message
object from a bytes object & The fields of the message object should have been
parsed properly \\
\hline
4 & Test parsing of invalid packet & Attempt to use the \_parse() method to
create a message object from a bytes object with an incorrect length & A
MessageException should be thrown \\
\bottomrule
\caption{Message Format Class Tests}
\label{table:msg-frmt-tests}
\end{longtable}

\subsubsection{Database Tests}

The following features are to be tested in order to test the integrity of the
database:

\begin{itemize}
    \item Control server populating right tables 
    \item Proper formatted data uploaded to appropriate fields
    \item Records of the temperature readings of all employees are accessible
          for at least 14 days
\end{itemize}

\paragraph{Scenario 1}
Populating sample test data access\_entry table.

\noindent
Features to be tested:
\begin{itemize}
    \item Control server populating right tables
\end{itemize}

\noindent
Initial setup: set up Raspberry pi connected to monitor keyboard and mouse.
Open command prompt on Raspberry Pi.

\noindent
Test scenario steps:
\begin{enumerate}
    \item Invoke SQLite by typing this command in the command prompt
\begin{lstlisting}
sqlite3
\end{lstlisting}
    \item Open the database by typing this command 
\begin{listing}[H]
\begin{minted}[]{text}
.open project_database.db
\end{minted}
\end{listing}
    \item Populate data to employee access\_entry table
\begin{listing}[H]
\begin{minted}[]{sql}
INSERT INTO access_entry
    VALUES(19234,time('now'),date('now'), "S_entry",
           "Yes", 38.0, "Authorized");
INSERT INTO access_entry
    VALUES(19437,time('now','-12 minutes'),date('now'),
          "E_entry", "Yes", 38.3, "Authorized");
INSERT INTO access_entry
    VALUES(19632,time('now','-1 hour'),date('now'),
           "N_entry", "Yes", 39.2, "Unauthorized");
INSERT INTO access_entry
    VALUES(19957,time('now','-5 minutes'),date('now'),
           "W_entry", "Yes", 38.5, "Authorized");
\end{minted}
\end{listing}
    \item Query data using \lstinline{Select*} command
\begin{listing}[H]
\begin{minted}[]{sql}
SELECT * FROM access_entry;
\end{minted}
\end{listing}
\end{enumerate}

\noindent
Expected test scenario result: table populated with test sample data and query
outputs.

\begin{lstlisting}
19234 | 09:30:23 | 2020-10-26 | S_entry | Yes | 38.0 | Authorized
19437 | 09:18:23 | 2020-10-26 | E_entry | Yes | 38.3 | Authorized
19632 | 08:30:23 | 2020-10-26 | N_entry | Yes | 39.2 | Unauthorized
19957 | 09:25:23 | 2020-10-26 | W_entry | Yes | 38.5 | Authorized
\end{lstlisting}

\paragraph{Scenario 2}
Populating sample test data nfc\_and\_employer\_id table.

\noindent
Features to be tested:
\begin{itemize}
    \item Proper formatted data uploaded to appropriate fields
\end{itemize}

\noindent
Initial setup: set up Raspberry pi connected to monitor keyboard and mouse.
Open command prompt on Raspberry Pi.

\noindent
Test scenario steps:
\begin{enumerate}
    \item Invoke SQLite by typing this command in the command prompt
\begin{listing}[H]
\begin{minted}[]{bash}
sqlite3
\end{minted}
\end{listing}
    \item Open the database by typing this command
\begin{listing}[H]
\begin{minted}[]{sql}
.open project_database.db
\end{minted}
\end{listing}
    \item Populate data to employee access\_entry table
\begin{listing}[H]
\begin{minted}[]{sql}
INSERT INTO access_entry
    VALUES(19234,time('now'),date('now'), "S_entry",
           "Yes", 38.0, "Authorized");
INSERT INTO access_entry
    VALUES(19437,time('now','-12 minutes'),date('now'),
           "E_entry","Yes", 38.3, "Authorized");
INSERT INTO access_entry
    VALUES(19632,time('now','-1 hour'),date('now'),
           "N_entry","Yes", 39.2, "Unauthorized");
INSERT INTO access_entry
    VALUES(19957,time('now','-5 minutes'),date('now'),
           "W_entry","Yes", 38.5, "Authorized");
\end{minted}
\end{listing}
    \item Query data using \lstinline{Select*} command
\begin{listing}[H]
\begin{minted}[]{sql}
SELECT access_time, access_date FROM access_entry;
\end{minted}
\end{listing}
\end{enumerate}

\noindent
Expected test scenario result: table populated with test sample data and query
outputs.

\begin{listing}[H]
\begin{minted}[]{text}
09:30:23 | 2020-10-26
09:18:23 | 2020-10-26
08:30:23 | 2020-10-26
09:25:23 | 2020-10-26
\end{minted}
\end{listing}

\paragraph{Scenario 3}
Populating sample test data access\_entry table.

\noindent
Features to be tested:
\begin{itemize}
    \item Records of the temperature readings of all employees are accessible
          for at least 14 days
\end{itemize}

\noindent
Initial setup: set up Raspberry pi connected to monitor keyboard and mouse.
Open command prompt on Raspberry Pi.

\noindent
Test scenario steps:
\begin{enumerate}
    \item Invoke SQLite by typing this command in the command prompt
\begin{listing}[H]
\begin{minted}[]{bash}
sqlite3
\end{minted}
\end{listing}
    \item Open the database by typing this command
\begin{listing}[H]
\begin{minted}[]{sql}
.open project_database.db
\end{minted}
\end{listing}
    \item Populate data to employee access\_entry table
\begin{listing}[H]
\begin{minted}[]{sql}
INSERT INTO access_entry
    VALUES (19234, time('now'), date('now'), "S_entry",
            "Yes", 38.0, "Authorized");
INSERT INTO access_entry
    VALUES (19437, time('now','-12 minutes'),
            date('now', '-1 day'), "E_entry", "Yes", 38.3,
            "Authorized");
INSERT INTO access_entry
    VALUES (19632, time('now','-1 hour'),
            date('now', '-3 days'), "N_entry", "Yes", 39.2,
            "Unauthorized");
INSERT INTO access_entry
    VALUES (19957, time('now','-5 minutes'),
            date('now', '-7 days'), "W_entry", "Yes", 38.5,
            "Authorized");
INSERT INTO access_entry
    VALUES (19234, time('now'), date('now', '-13 days'),
            "S_entry", "Yes", 38.0, "Authorized");
INSERT INTO access_entry
    VALUES (19437, time('now','-12 minutes'),
            date('now', '-14 days'), "E_entry", "Yes", 38.3,
            "Authorized");
INSERT INTO access_entry
    VALUES (19632, time('now','-1 hour'),
            date('now', '-14 days'), "N_entry", "Yes", 39.2,
            "Unauthorized");
INSERT INTO access_entry
    VALUES (19957, time('now','-5 minutes'),
            date('now', '-15 days'), "W_entry", "Yes", 38.5,
            "Authorized");
\end{minted}
\end{listing}
    \item Query data using \lstinline{Select*} command
\begin{listing}[H]
\begin{minted}[]{sql}
SELECT * FROM access_entry WHERE access_date > date('now');
\end{minted}
\end{listing}
\end{enumerate}

\noindent
Expected test scenario result: table populated with test sample data and query
outputs.

\begin{listing}[H]
\begin{minted}[]{text}
19437 | 09:18:23 | 2020-10-25 | E_entry | Yes | 38.3 | Authorized
19632 | 08:30:23 | 2020-10-23 | N_entry | Yes | 39.2 | Unauthorized
19957 | 09:25:23 | 2020-10-19 | W_entry | Yes | 38.5 | Authorized
19234 | 09:30:23 | 2020-10-13 | S_entry | Yes | 38.0 | Authorized
19437 | 09:18:23 | 2020-10-12 | E_entry | Yes | 38.3 | Authorized
19632 | 08:30:23 | 2020-10-12 | N_entry | Yes | 39.2 | Unauthorized
19957 | 09:25:23 | 2020-10-11 | W_entry | Yes | 38.5 | Authorized
\end{minted}
\end{listing}

\subsubsection{GUI Tests}

\paragraph{Generating a settings file}

Table \ref{table:op-gui-settings-tests} shows the test cases that will be used to test the
ability of the GUI to generate a settings file.  Each test will have a known
proper settings file that is manually created.  If the test is able to generate
a file equivalent to the known proper settings file, the test passes.

\begin{longtable}[htb]{>{\centering\arraybackslash}m{0.75cm}|>{\centering\arraybackslash}m{4cm}|>{\centering\arraybackslash}m{4.5cm}|>{\centering\arraybackslash}m{4cm}}
\toprule
Test & Description & Setup & Expected Result \\
\midrule
1 & Test default settings & Have default settings loaded & Generated
settings file matches known good settings file \\
\hline
2 & Set temperature range to 35-37.5 & Have default settings loaded & Generated
settings file matches known good settings file \\
\hline
3 & Set max users to 100 & Have default settings loaded & Generated
settings file matches known good settings file \\
\bottomrule
\caption{Operator GUI Database Query Tests}
\label{table:op-gui-settings-tests}
\end{longtable}

\paragraph{Querying the database}

Table \ref{table:op-gui-db-query-tests} shows the test cases that will be used to test the
ability of the GUI to query a database. The GUI will be tested using the same
test database each time.

\begin{longtable}[htb]{>{\centering\arraybackslash}m{0.75cm}|>{\centering\arraybackslash}m{4cm}|>{\centering\arraybackslash}m{4.5cm}|>{\centering\arraybackslash}m{4cm}}
\toprule
Test & Description & Setup & Expected Result \\
\midrule
1 & Query 100 latest accesses & Have test database setup & 100 latest accesses
are shown correctly and formatted in an appealing way \\
\hline
2 & Query specific user & Have test database setup & All accesses from a specific user
are shown correctly and formatted in an appealing way \\
\hline
3 & Query successful accesses & Have test database setup & All succesful accesses
are shown correctly and formatted in an appealing way \\
\hline
4 & Query failed accesses & Have test database setup & All failed accesses
are shown correctly and formatted in an appealing way \\
\bottomrule
\caption{Operator GUI Database Query Tests}
\label{table:op-gui-db-query-tests}
\end{longtable}

\subsection{Final Demo}

Table \ref{table:final-tests} lists the scenarios that we will test during the
final demo. These scenarios have been chosen to demonstrate the functional
requirements as listed in section \ref{sec:problem-statement}.

% Functional requirements:
% - Control access to a building using security badges.
% - Require users to present their security badge when they enter or exit the
%   building in order to track the number of users in the building.
% - Measure user's temperatures when they are entering the building in order to
%   determine if they have possible symptoms.
% - The door node should have a range sensor to determine whether users are in
%   an appropriate position for a temperature reading.
% - Do not allow more users to enter the building if the a preset maximum
%   capacity has been reached.
% - An multicoloured LED at each door node should indicate be used to indicate
%   the status of the door node. The LED should be normally red when the door is
%   locked and should change to green when the door is unlocked. The LED should
%   be orange when in the process of taking a temperature reading if the user is
%   not within an appropriate range of the temperature sensor.


\begin{longtable}[htb]{>{\centering\arraybackslash}m{3cm}|>{\centering\arraybackslash}m{3.5cm}|>{\centering\arraybackslash}m{3cm}|>{\centering\arraybackslash}m{3.5cm}}
\toprule
Description & Requirement & Procedure & Expected Result \\
\midrule
Ideal Building Entry & Control access to a building using security badges,
measure users' temperatures when they are entering the building, change LED 
colours & Present a valid security badge to the incoming reader, provide a
normal temperature reading the temperature sensor & The electronic door lock
should be actuated to allow entry, the access should be logged in the database,
the count of people in the building should incremented and the LED at the door
node should change to green temporarily \\
\hline
Ideal Building Exit & Require users to present their security badge when exiting
the building in order to track the number of users in the building & Present a
valid building security badge to the outgoing reader at a door node & The door
node should be actuated to allow the user to exit, the exit should be logged in
the database and the count of people in the building should be decremented \\
\hline
Attempted building entry with invalid security badge & Control access to a
building using security badges & Present an invalid security badge to the
incoming reader & The electronic door lock should not be actuated and entry
should not be allowed, the attempted access should be logged in the database and
the LED at the door node should remain red \\
\hline
Attempted building entry with fever & Measure user's temperature when they are
entering the building & Present a valid security badge to the reader but then
present a temperature that is too high & The electronic door lock should not be
actuated and entry should not be allowed, the attempted access should be logged
in the database and the LED at the door node should remain red \\
\hline
Attempted building entry with low temperature & Measure user's temperature when
they are entering the building & Present a valid security badge to the reader
but then present a temperature that is too low & The electronic door lock should
not be actuated and entry should not be allowed, the attempted access should be
logged in the database and the LED at the door node should remain red \\
\hline
Attempted building entry when building is at capacity & Do not allow users to
enter the building if a maximum capacity has been reached & Set the maximum
capacity of the system to one users, enter with a valid security badge and
temperature reading then attempt to enter with a second valid security badge &
The electronic door lock should not be actuated and entry should not be allowed,
the attempted access should be logged in the database and the LED at the door
node should remain off \\
\hline
Temperature measurement from incorrect distance & Determine whether users are in
an appropriate position for a temperature reading & Present a valid security
badge to the incoming reader but initially stand approximately one meter away
from the temperature sensor, slowly move closer until a temperature reading is
taken & The LED should be illuminated orange and the electronic door lock should
not be actuated until the user is within an appropriate range of the infrared
temperature sensor \\
\hline
View access logs & Access logs should be visible to users using the operator GUI
& Open the operator GUI and view access logs & Access logs should be visible,
accurate and up to date \\
\bottomrule
\caption{Final Demo Tests}
\label{table:final-tests}
\end{longtable}

