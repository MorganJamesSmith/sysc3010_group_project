\subsection{End to End Demo}

\noindent
Software requirements:
\begin{itemize}
    \item Set up Thingspeak channel 
    \item Hard code sensor signal communication to Thingspeak 
    \item Have test stubs which emulate the functionality of the hardware
          components ready
\end{itemize}

\subsubsection*{Entry Access}

\paragraph{Scenario 1}
Building accepting entries, employee with valid NFC card, authorized status,
normal body temperature, ideal distance away from temperature sensor and
building accessible after entry.

\noindent
Features to be tested:
\begin{itemize}
    \item Communication between control server and Thingspeak
    \item Communication between door node and Thingspeak
\end{itemize}

\noindent
Test scenario steps:
\begin{enumerate}
    \item Control server checks for the number of people in the building and 
          sends an DOOR\_STATE\_UPDATE to Thingspeak channel
    \item Door node receives the DOOR\_STATE\_UPDATE from Thingspeak channel and 
          if available, displays green light to indicate door node accepting 
          entries on the LED display
    \item NFC reader identifies NFC id card being tapped and sends 
          ACCESS\_REQUEST with card id to Thingspeak channel
    \item Control server receives ACCESS\_REQUEST from Thingspeak channel and 
          validates NFC card id and identifies employee id by querying the 
          database for the NFC card id 
    \item Once acquiring the valid employee id, the control server queries the 
          database for recent employee status, access type and validity to 
          consider entry access
    \item If recent status is authorized, access type was exit and validity is
          less than 3, control server determines current access type to be 
          entry and sends INFROMATION\_REQUEST to determine if employee in 
          distance range to Thingspeak channel
    \item Door node receives INFROMATION\_REQUEST from the Thingspeak channel
          and activates the LED display and displays orange to indicate that the
          access entry process is in progress 
    \item Door node activates the rangefinder to identify users range from
          temperature sensor
    \item Door node records the range value and sends an INFORMATION\_RESPONSE
          to the Thingspeak channel
    \item Control server receives the INFORMATION\_RESPONSE and compares the
          received range value with the ideal range value
    \item If in ideal range, control server sends an INFORMATION\_REQUEST to
          measure the user’s body temperature
    \item Door node receives the INFORMATION\_REQUEST and activates the 
          temperature sensor to record the user’s body temperature 
    \item Door node then sends an INFROMATION\_RESPONSE with the recorded
          temperature measurement to the Thingspeak channel 
    \item Control server receives the INFORMATION\_RESPONSE from the Thingspeak
          channel
    \item Control server compares recorded temperature value to acceptable
          temperature range and determines the status of the entry request
    \item If in acceptable temperature range, control server updates the status
          of the employee to “authorized” and saves all the information for
          access entry in the database and sets the validity field to 0
    \item Control server increments the number of people in the building by 1
    \item Control server sends an ACCESS\_RESPONSE to the Thingspeak channel
    \item Control server re-evaluates the number of people in the building and
          sends a DOOR\_STATE\_UPDATE based on the accessibility status of the 
          building to the Thingspeak channel
    \item Door node receives the ACCESS\_REPONSE and DOOR\_STATE\_UPDATE from 
          the Thingspeak channel
    \item From ACCESS\_RESPONSE received by the door node, if access is
          authorized, door node activates LED display to display a flashing
          green light to indicate employee’s access has been authorized 
    \item Door node also unlocks the electronic lock for 30 seconds
    \item After 30 sec, Door node locks the electronic lock
    \item From the DOOR\_STATE\_UPDATE received by the door node, if the
          building is accessible, door node activates LED display to display a
          green light to indicate system is accepting entry
\end{enumerate}

\noindent
Expected test scenario result: proper communication between nodes and control
server occurred to authorize employee access to the workplace.


\paragraph{Scenario 2}
Building accepting entries, employee with valid NFC card, authorized status,
normal body temperature, ideal distance away from temperature sensor and
building not accessible after entry

\noindent
Features to be tested:
\begin{itemize}
    \item Communication between control server and Thingspeak
    \item Communication between door node and Thingspeak
\end{itemize}

\noindent
Test scenario steps:
\begin{enumerate}
    \item Control server checks for the number of people in the building and
          sends a DOOR\_STATE\_UPDATE to Thingspeak channel
    \item Door node receives the DOOR\_STATE\_UPDATE from Thingspeak channel and
          if available, displays green light to indicate door node accepting
          entries on the LED display
    \item NFC reader identifies NFC id card being tapped and sends
          ACCESS\_REQUEST with card id to Thingspeak channel
    \item Control server receives ACCESS\_REQUEST from Thingspeak channel and
          validates NFC card id and identifies employee id by querying the
          database for the NFC card id 
    \item Once acquiring the valid employee id, the control server queries the
          database for recent employee status, access type and validity to
          consider entry access
    \item If recent status is authorized, access type was exit and validity is
          less than 3, control server determines current access type to be entry
          and sends INFROMATION\_REQUEST to determine if employee in distance
          range to Thingspeak channel
    \item Door node receives INFROMATION\_REQUEST from the Thingspeak channel
          and activates the LED display and displays orange to indicate that the
          access entry process is in progress 
    \item Door node activates the rangefinder to identify users range from
          temperature sensor
    \item Door node records the range value and sends an INFORMATION\_RESPONSE
          to the Thingspeak channel
    \item Control server receives the INFORMATION\_RESPONSE and compares the
          received range value with the ideal range value
    \item If in ideal range, control server sends an INFORMATION\_REQUEST to
          measure the user’s body temperature
    \item Door node receives the INFORMATION\_REQUEST and activates the
          temperature sensor to record the user’s body temperature 
    \item Door node then sends an INFROMATION\_RESPONSE with the recorded
          temperature measurement to the Thingspeak channel 
    \item Control server receives the INFORMATION\_RESPONSE from the Thingspeak
          channel
    \item Control server compares recorded temperature value to acceptable
          temperature range and determines the status of the entry request
    \item If in acceptable temperature range, control server updates the status
          of the employee to “authorized” and saves all the information for
          access entry in the database and sets the validity field to 0
    \item Control server increments the number of people in the building by 1
    \item Control server sends an ACCESS\_RESPONSE to the Thingspeak channel
    \item Control server re-evaluates the number of people in the building and
          sends a DOOR\_STATE\_UPDATE based on the accessibility status of the
          building to the Thingspeak channel
    \item Door node receives the ACCESS\_REPONSE and DOOR\_STATE\_UPDATE from
          the Thingspeak channel
    \item From ACCESS\_RESPONSE received by the door node, if access is
          authorized, door node activates LED display to display a flashing
          green light to indicate employee’s access has been authorized 
    \item Door node also unlocks the electronic lock for 30 seconds
    \item After 30 sec, Door node locks the electronic lock
    \item From the DOOR\_STATE\_UPDATE received by the door node, if the
          building is not accessible, door node activates LED display to display
          a red light to indicate system is not accepting entry
\end{enumerate}

\noindent
Expected test scenario result: Proper communication between door node and
control server occurred to authorize employee access to the workplace and
restrict entry to additional employees after the process.

\paragraph{Scenario 3}
Building accepting entries, employee with invalid NFC card, authorized status,
normal body temperature, in ideal distance range away from temperature sensor at
entry node and building accessible after entry attempt.

\noindent
Features to be tested:
\begin{itemize}
    \item Communication between control server and Thingspeak
    \item Communication between door node and Thingspeak
\end{itemize}

\noindent
Test scenario steps:
\begin{enumerate}
    \item Control server checks for the number of people in the building and
          sends an DOOR\_STATE\_UPDATE to Thingspeak channel
    \item Door node receives the DOOR\_STATE\_UPDATE from Thingspeak channel and
         if available, displays green light to indicate door node accepting
         entries on the LED display
    \item NFC reader identifies NFC id card being tapped and sends
          ACCESS\_REQUEST with card id to Thingspeak channel
    \item Control server receives ACCESS\_REQUEST from Thingspeak channel and
          queries the database for the NFC card id to be associated to the
          employee id
    \item If no valid employee id acquired, the control server sends an
          ACCESS\_RESPONSE indicating invalid id card being scanned to the
          Thingspeak channel
    \item Control server re-evaluates the number of people in the building and
          sends a DOOR\_STATE\_UPDATE based on the accessibility status of the
          building to the Thingspeak channel
    \item Door node receives ACCESS\_RESPONSE and DOOR\_STATE\_UPDATE from the
          Thingspeak channel 
    \item From the ACCESS\_REPONSE received, door node locks the electronic
          lock and displays flashing red light to indicate that the access entry
          attempt is invalid
    \item From the DOOR]\_STATE\_UPDATE received by the door node, if the
          building is accessible, door node activates LED display to display a
          green light to indicate system is accepting entry
\end{enumerate}

\noindent
Expected test scenario result: proper communication between nodes and control
server occurred to unauthorize employee access to workplace.

\paragraph{Scenario 4}
Building accepting entries, employee with valid NFC card, authorized status,
normal body temperature, not in ideal distance range away from temperature
sensor and building accessible after entry attempt.

\noindent
Features to be tested:
\begin{itemize}
    \item Communication between control server and Thingspeak
    \item Communication between door node and Thingspeak
\end{itemize}

\noindent
Test scenario steps:
\begin{enumerate}
    \item Control server checks for the number of people in the building and
          sends an DOOR\_STATE\_UPDATE to Thingspeak channel
    \item Door node receives the DOOR\_STATE\_UPDATE from Thingspeak channel and
          if available, displays green light to indicate door node accepting
          entries on the LED display
    \item NFC reader identifies NFC id card being tapped and sends
          ACCESS\_REQUEST with card id to Thingspeak channel
    \item Control server receives ACCESS\_REQUEST from Thingspeak channel and
          validates NFC card id and identifies employee id by querying the
          database for the NFC card id 
    \item Once acquiring the valid employee id, the control server queries the
          database for recent employee status, access type and validity to
          consider entry access
    \item If recent status is authorized, access type was exit and validity is
          less than 3, control server determines current access type to be entry
          and sends INFROMATION\_REQUEST to determine if employee in distance
          range to Thingspeak channel and increments validity by 1
    \item Door node receives INFROMATION\_REQUEST from the Thingspeak channel
          and activates the LED display and displays orange to indicate that
          the access entry process is in progress 
    \item Door node activates the rangefinder to identify users range from
          temperature sensor
    \item Door node records the range value and sends an INFORMATION\_RESPONSE
          to the Thingspeak channel
    \item Control server receives the INFORMATION\_RESPONSE and compares the
          received range value with the ideal range value
    \item Since not in ideal range, control server checks the validity field to
          see if its less than 3, if yes then it sends an INFORMATION\_REQUEST
          to check again if employee in range and increments the validity by 1 
    \item Door node receives the INFORMATION\_REQUEST from the Thingspeak
          channel and activates the rangefinder again to identify if employee in
          range of temperature sensor
    \item Door node sends an INFORMATION\_RESPONSE with the recorded range value
    \item Control server receives the INFORMATION\_RESPONSE from the Thingspeak
          channel and compares the received range value with the ideal range
          value
    \item Since not in ideal range, control server checks the validity field to
          see if its less than 3, if yes then it sends an INFORMATION\_REQUEST
          to check again if employee in range and increments the validity by 1 
    \item Door node receives the INFORMATION\_REQUEST and activates the
          rangefinder again to identify if employee in range of temperature
          sensor
    \item Door node sends an INFORMATION\_RESPONSE with recorded range value to
          the Thingspeak channel
    \item Control server receives the INFORMATION\_RESPONSE from the Thingspeak
          channel and compares the received range value with the ideal range
          value 
    \item Since not in ideal range, control server checks the validity field to
          see if its less than 3, if not less than 3 then control serveri
          determines the access entry is invalid and determines the
          ACCESS\_RESPONSE to be sent
    \item control server updates the validity field for the access entry to 3
          and saves all the information for access entry in the database
    \item Control server re-evaluates the number of people in the building and
          sends a DOOR\_STATE\_UPDATE based on the accessibility status of the
          building to the Thingspeak channel
    \item Door node receives the ACCESS\_REPONSE and DOOR\_STATE\_UPDATE from
          the Thingspeak channel
    \item From ACCESS\_RESPONSE received by the door node, since access is
          invalid, door node locks the electronic lock and displays a red light
          to indicate access denied 
    \item From the DOOR\_STATE\_UPDATE received by the door node, if the
          building is accessible, door node activates LED display to display a
          green light to indicate system is accepting entry
\end{enumerate}

\noindent
Expected test scenario result: Proper communication between door node and
control server occurred to let the employee know to stand in an ideal distance
range away from the temperature sensor.

\paragraph{Scenario 5}
Building accepting entries, employee with valid NFC card, authorized status, in
ideal distance range away from temperature sensor, body temperature indicating
fever and building accessible after entry attempt.

\noindent
Features to be tested:
\begin{itemize}
    \item Communication between control server and Thingspeak
    \item Communication between door node and Thingspeak
\end{itemize}

\noindent
Test scenario steps:
\begin{enumerate}
    \item Control server checks for the number of people in the building and
          sends an DOOR\_STATE\_UPDATE to Thingspeak channel
    \item Door node receives the DOOR\_STATE\_UPDATE from Thingspeak channel and
          if available, displays green light to indicate door node accepting
          entries on the LED display
    \item NFC reader identifies NFC id card being tapped and sends
          ACCESS\_REQUEST with card id to Thingspeak channel
    \item Control server receives ACCESS\_REQUEST from Thingspeak channel and
          validates NFC card id and identifies employee id by querying the
          database for the NFC card id 
    \item Once acquiring the valid employee id, the control server queries the
          database for recent employee status, access type and validity to
          consider entry access
    \item If recent status is authorized, access type was exit and validity is
          less than 3, control server determines current access type to be entry
          and sends INFROMATION\_REQUEST to determine if employee in distance
          range to Thingspeak channel
    \item Door node receives INFROMATION\_REQUEST from the Thingspeak channel
          and activates the LED display and displays orange to indicate that the
          access entry process is in progress 
    \item Door node activates the rangefinder to identify users range from
          temperature sensor
    \item Door node records the range value and sends an INFORMATION\_RESPONSE
          to the Thingspeak channel
    \item Control server receives the INFORMATION\_RESPONSE and compares the
          received range value with the ideal range value
    \item If in ideal range, control server sends an INFORMATION\_REQUEST to
          measure the user’s body temperature
    \item Door node receives the INFORMATION\_REQUEST and activates the
          temperature sensor to record the user’s body temperature 
    \item Door node then sends an INFROMATION\_RESPONSE with the recorded
          temperature measurement to the Thingspeak channel 
    \item Control server receives the INFORMATION\_RESPONSE from the Thingspeak
          channel
    \item Control server compares recorded temperature value to acceptable
          temperature range and determines the status of the entry request
    \item If not in acceptable temperature range, control server checks the
          validity field to see if its less than 3, if yes then it sends an
          INFORMATION\_REQUEST to measure the user body temperature again and
          increments the validity by 1 
    \item Door node receives the INFORMATION\_REQUEST from the Thingspeak
          channel and activates the temperature sensor again to record user body
          temperature
    \item Door node sends an INFORMATION\_RESPONSE with the recorded body
          temperature value
    \item Control server receives the INFORMATION\_RESPONSE from the Thingspeak
          channel and compares the received temperature value with the ideal
          temperature value
    \item Since not in ideal range, control server checks the validity field to
          see if its less than 3, if yes then it sends an INFORMATION\_REQUEST
          to record user body temperature again and increments the validity by 1 
    \item Door node receives the INFORMATION\_REQUEST and activates the
          temperature sensor again to record body temperature of user
    \item Door node sends an INFORMATION\_RESPONSE with recorded temperature
          value to the Thingspeak channel
    \item Control server receives the INFORMATION\_RESPONSE from the Thingspeak
          channel and compares the received temperature value with the ideal
          temperature value 
    \item Since not in ideal range, control server checks the validity field to
          see if its less than 3, if not less than 3 then control server
          determines the access entry is invalid and determines the
          ACCESS\_RESPONSE to be sent
    \item control server updates the validity field for the access entry to 3,
          updates the status to unauthorized and saves all the information for
          access entry in the database
    \item Control server re-evaluates the number of people in the building and
          sends a DOOR\_STATE\_UPDATE based on the accessibility status of the
          building to the Thingspeak channel
    \item Door node receives the ACCESS\_REPONSE and DOOR\_STATE\_UPDATE from
          the Thingspeak channel
    \item From ACCESS\_RESPONSE received by the door node, since access is
          unauthorized, door node locks the electronic lock and displays a
          flashing red light to indicate access denied 
    \item From the DOOR\_STATE\_UPDATE received by the door node, if the
          building is accessible, door node activates LED display to display a
          green light to indicate system is accepting entry
\end{enumerate}

\noindent
Expected test scenario result: Proper communication between nodes and control
server occurred to unauthorize employee and restrict access to workplace.

\paragraph{Scenario 6}
Building not accepting entries, employee with invalid NFC card, authorized
status, in ideal distance range, normal body temperature and building accessible
after entry attempt.

\noindent
Features to be tested:
\begin{itemize}
    \item Communication between control server and Thingspeak
    \item Communication between door node and Thingspeak
\end{itemize}

\noindent
Test scenario steps:
\begin{enumerate}
    \item Control server checks for the number of people in the building and
          sends a DOOR\_STATE\_UPDATE to Thingspeak channel
    \item Door node receives the DOOR\_STATE\_UPDATE from Thingspeak channel and
          since entry not available, displays red light to indicate door node
          not accepting entries on the LED display
    \item NFC reader identifies NFC id card being tapped and sends
          ACCESS\_REQUEST with card id to Thingspeak channel
    \item Control server receives ACCESS\_REQUEST from Thingspeak channel and
          validates NFC card id and identifies employee id by querying the
          database for the NFC card id 
    \item Once acquiring the valid employee id, the control server queries the
          database for recent employee status, access type and validity to
          consider entry access
    \item If recent status is authorized, access type was exit and validity is
          less than 3, control server determines current access type to be entry
          and sends INFROMATION\_REQUEST to determine if employee in distance
          range to Thingspeak channel
    \item Control server receives ACCESS\_REQUEST from Thingspeak channel and
          since building not the control server sends an ACCESS\_RESPONSE
          indicating entry not available
    \item Control server re-evaluates the number of people in the building and
          sends a DOOR\_STATE\_UPDATE based on the accessibility status of the
          building to the Thingspeak channel
    \item Door node receives ACCESS\_RESPONSE and DOOR\_STATE\_UPDATE from the
          Thingspeak channel 
    \item From the ACCESS\_REPONSE received, door node locks the electronic lock
          and displays red light to indicate building not accepting entries 
    \item From the DOOR\_STATE\_UPDATE received by the door node, since building
          still not accessible, door node activates LED display to display a red
          light to indicate system is not accepting entry
\end{enumerate}

\noindent
Expected test scenario result: Proper communication between door node and
control server occurred to let the employee know entry not available.

\paragraph{Scenario 7}
Building accepting entries, employee with invalid NFC card, unauthorized status 
(over 14 days ago), in ideal distance range away from temperature sensor, normal
body temperature and building accessible after entry attempt.

\noindent
Features to be tested:
\begin{itemize}
    \item Communication between control server and Thingspeak
    \item Communication between door node and Thingspeak
\end{itemize}

\noindent
Test scenario steps:
\begin{enumerate}
    \item Control server checks for the number of people in the building and
          sends a DOOR\_STATE\_UPDATE to Thingspeak channel
    \item Door node receives the DOOR\_STATE\_UPDATE from Thingspeak channel and
          if available, displays green light to indicate door node accepting
          entries on the LED display
    \item NFC reader identifies NFC id card being tapped and sends
          ACCESS\_REQUEST with card id to Thingspeak channel
    \item Control server receives ACCESS\_REQUEST from Thingspeak channel and
          validates NFC card id and identifies employee id by querying the
          database for the NFC card id 
    \item Once acquiring the valid employee id, the control server queries the
          database for recent employee status, access type and validity to
          consider entry access
    \item If recent status is unauthorized, access type was entry and validity
          is 3, control server determines current access type to be entry and
          queries the database for date of most recent entry attempt
    \item If most recent entry attempt date over 14 days ago, control server
          sends INFROMATION\_REQUEST to determine if employee in distance range
          to Thingspeak channel
    \item Door node receives INFROMATION\_REQUEST from the Thingspeak channel
          and activates the LED display and displays orange to indicate that the
          access entry process is in progress 
    \item Door node activates the rangefinder to identify users range from
          temperature sensor
    \item Door node records the range value and sends an INFORMATION\_RESPONSE 
          to the Thingspeak channel
    \item Control server receives the INFORMATION\_RESPONSE and compares the
          received range value with the ideal range value
    \item If in ideal range, control server sends an INFORMATION\_REQUEST to
          measure the user’s body temperature
    \item Door node receives the INFORMATION\_REQUEST and activates the
          temperature sensor to record the user’s body temperature 
    \item Door node then sends an INFROMATION\_RESPONSE with the recorded
          temperature measurement to the Thingspeak channel 
    \item Control server receives the INFORMATION\_RESPONSE from the Thingspeak
          channel
    \item Control server compares recorded temperature value to acceptable
          temperature range and determines the status of the entry request
    \item If in acceptable temperature range, control server updates the status
          of the employee to “authorized” and saves all the information for
          access entry in the database and sets the validity field to 0
    \item Control server increments the number of people in the building by 1
    \item Control server sends an ACCESS\_RESPONSE to the Thingspeak channel
    \item Control server re-evaluates the number of people in the building and
          sends a DOOR\_STATE\_UPDATE based on the accessibility status of the
          building to the Thingspeak channel
    \item Door node receives the ACCESS\_REPONSE and DOOR\_STATE\_UPDATE from
          the Thingspeak channel
    \item From ACCESS\_RESPONSE received by the door node, if access is
          authorized, door node activates LED display to display a flashing
          green light to indicate employee’s access has been authorized 
    \item Door node also unlocks the electronic lock for 30 seconds
    \item After 30 sec, Door node locks the electronic lock
    \item From the DOOR\_STATE\_UPDATE received by the door node, if the
          building is accessible, door node activates LED display to display a
          green light to indicate system is accepting entry
\end{enumerate}

\noindent
Expected test scenario result: Proper communication between door node and
control server occurred to authorize employee access to the workplace.

\paragraph{Scenario 8}
Building accepting entries, employee with invalid NFC card, unauthorized status
(not over 14 days ago), in ideal distance range away from temperature sensor,
normal body temperature and building accessible after entry attempt.

\noindent
Features to be tested:
\begin{itemize}
    \item Communication between control server and Thingspeak
    \item Communication between door node and Thingspeak
\end{itemize}

\noindent
Test scenario steps:
\begin{enumerate}
    \item Control server checks for the number of people in the building and
          sends an DOOR\_STATE\_UPDATE to Thingspeak channel
    \item Door node receives the DOOR\_STATE\_UPDATE from Thingspeak channel and
          if available, displays green light to indicate door node accepting
          entries on the LED display
    \item NFC reader identifies NFC id card being tapped and sends
          ACCESS\_REQUEST with card id to Thingspeak channel
    \item Control server receives ACCESS\_REQUEST from Thingspeak channel and
          validates NFC card id and identifies employee id by querying the
          database for the NFC card id 
    \item Once acquiring the valid employee id, the control server queries the
          database for recent employee status, access type and validity to
          consider entry access
    \item If recent status is unauthorized, access type was entry and validity
          is 3, control server determines current access type to be entry and
          queries the database for date of most recent entry attempt
    \item If most recent entry attempt date not over 14 days ago, control server
          sends ACCESS\_RESPONSE indicated access unauthorized to the Thingspeak
          channel and no information for the entry attempt is saved in the
          database
    \item Control server re-evaluates the number of people in the building and
          sends a DOOR\_STATE\_UPDATE based on the accessibility status of the
          building to the Thingspeak channel
    \item Door node receives the ACCESS\_REPONSE and DOOR\_STATE\_UPDATE from
          the Thingspeak channel
    \item From ACCESS\_RESPONSE received by the door node, since access is
          unauthorized, door node locks the electronic lock and displays a
          flashing red light to indicate access denied 
    \item From the DOOR\_STATE\_UPDATE received by the door node, if the
          building is accessible, door node activates LED display to display a
          green light to indicate system is accepting entry
\end{enumerate}

\noindent
Expected test scenario result: Proper communication between door node and
control server occurred to let the employee know invalid access entry attempt.

\paragraph{Scenario 9}
Employee with valid NFC card and at exit node.

\noindent
Features to be tested:
\begin{itemize}
    \item Communication between control server and Thingspeak
    \item Communication between door node and Thingspeak
\end{itemize}

\noindent
Test scenario steps:
\begin{enumerate}
    \item NFC reader identifies NFC id card being tapped and sends
          ACCESS\_REQUEST with card id to Thingspeak channel
    \item Control server receives ACCESS\_REQUEST from Thingspeak channel and
          validates NFC card id and identifies employee id by querying the
          database for the NFC card id 
    \item Once acquiring the valid employee id, the control server queries the
          database for recent employee status, access type and validity toi
          consider entry access
    \item If recent status is authorized, access type was entry and validity is
          3 or less than 3, control server determines current access type to be
          exit 
    \item If most recent access type is entry attempt control server sends
          ACCESS\_RESPONSE indicated exit access to the Thingspeak channel, logs
          information for the exit attempt in the database and subtracts1 from
          the number of people in the building  
    \item Control server re-evaluates the number of people in the building and
          sends a DOOR\_STATE\_UPDATE based on the accessibility status of the
          building to the Thingspeak channel
    \item Door node receives the ACCESS\_REPONSE and DOOR\_STATE\_UPDATE from
          the Thingspeak channel
    \item From ACCESS\_RESPONSE received by the door node, since access type is
          exit, door node unlocks the electronic lock at exit node 
    \item From the DOOR\_STATE\_UPDATE received by the door node, if the
          building is accessible, door node activates LED display to display a
          green light to indicate system is accepting entry
\end{enumerate}

\noindent
Expected test scenario result:Employee successfully exits the building an
accessibility of building updated.   

\subsection{Testing Demo}

\subsubsection{Hardware Tests}

\paragraph{Infrared Temperature Sensor}

Table \ref{table:ir-tests} shows the test cases that will be used to test the
infrared temperature sensor.

\begin{longtable}[htb]{>{\centering\arraybackslash}m{0.75cm}|>{\centering\arraybackslash}m{4cm}|>{\centering\arraybackslash}m{4.5cm}|>{\centering\arraybackslash}m{4cm}}
\toprule
Test & Description & Setup & Expected Result \\
\midrule
1 & Measure a temperature within the normal range & Aim temperature sensor at
someone's forehead and take a measurement & Measured temperature value is within
the expected range \\
\hline
2 & Measure a temperature that could indicate a fever & Aim the temperature
sensor at a pot of water that has been heated to 39 ℃ and take a measurement &
Measured temperature value should be with 0.5 ℃ of 39 ℃. \\
\hline
3 & Measure a temperature below the expected range & Aim the temperature sensor
at any room temperature surface and take a measurement & The measurement should
be approximately room temperature. \\
\bottomrule
\caption{Infrared Temperature Sensor Tests}
\label{table:ir-tests}
\end{longtable}

\paragraph{NFC Security Badge Reader}

Table \ref{table:nfc-tests} shows the test cases that will be used to test the
NFC security badge reader.

\begin{longtable}[htb]{>{\centering\arraybackslash}m{0.75cm}|>{\centering\arraybackslash}m{4cm}|>{\centering\arraybackslash}m{4.5cm}|>{\centering\arraybackslash}m{4cm}}
\toprule
Test & Description & Setup & Expected Result \\
\midrule
1 & Green RGB LED functions & Have breadboard set up & Green is shown on LED \\
\hline
2 & Red RGB LED functions & Have breadboard set up & Red is shown on LED \\
\hline
3 & Green RGB LED can change to red & Have breadboard set up & Initially green
is shown on LED which then changes to red \\
\hline
4 & Red RGB LED can change to green & Have breadboard set up & Initially red is
shown on LED which then changes to green \\
\hline
5 & Red RGB LED can change to green & Have breadboard set up & Initially red is
shown on LED which then changes to being turned off \\
\hline
6 & MFRC 522 RFID Module activates & Have breadboard set up. Have a passive tag
ready to wave in front & Pi prints out data on RFID tag \\
\hline
7 & MFRC 522 RFID Module can note separate ID & Have breadboard set up. Have two
passive tags ready to wave in front & Pi prints out different data for each RFID
tag swiped in front. \\

\bottomrule
\caption{NFC Security Badge Reader Sensor Tests}
\label{table:nfc-tests}
\end{longtable}

\paragraph{Electronic Door Lock}

Table \ref{table:servo-tests} shows the test cases that will be used to test the
electronic door lock.

\begin{longtable}[htb]{>{\centering\arraybackslash}m{0.75cm}|>{\centering\arraybackslash}m{4cm}|>{\centering\arraybackslash}m{4.5cm}|>{\centering\arraybackslash}m{4cm}}
\toprule
Test & Description & Setup & Expected Result \\
\midrule
1 & Test servo Oscillation & Have breadboard set up & Servo alternates between
0 degrees and 180 degrees every second \\
\bottomrule
\caption{Electronic Door Lock Tests}
\label{table:servo-tests}
\end{longtable}

A test program will be written that will initially place the servo at an angle
of 0 degrees.  Then a loop will run that will move the servo arm to 180 degrees,
wait a second, move back to 0 degrees, and wait another second.  The operation
of the servo can be easily verified visually.

\paragraph{Time of Flight Range Sensor}

Table \ref{table:tof-tests} shows the test cases that will be used to test the
time of flight range sensor.

\begin{longtable}[htb]{>{\centering\arraybackslash}m{0.75cm}|>{\centering\arraybackslash}m{4cm}|>{\centering\arraybackslash}m{4.5cm}|>{\centering\arraybackslash}m{4cm}}
\toprule
Test & Description & Setup & Expected Result \\
\midrule
1 & Orange\_LED function activated & Have LED display set up with the Raspberry
Pi connected to monitor & RGB LED displays Orange \\
\hline
2 & Red\_LED function activated & Have LED display set up with the Raspberry Pi
connected to monitor & RGB LED displays Red \\
\hline
3 & Switch between orange and red LEDs & Have LED display set up with the
Raspberry Pi connected to monitor & RGB LED initially displays orange and then
switches to display red \\
\hline
4 & Switch between red and orange LEDs & Have LED display set up with the
Raspberry Pi connected to monitor & RGB LED initially displays red and then
switches to display orange \\
\hline
5 & VL6180X module activated & Have sensor connected to the Raspberry Pi and
connect Raspberry Pi to a display & Display prints out initial reading \\
\hline
6 & VL6180X module activated and Range\_Find class used to display multiple
range recordings & Have sensor connected to the Raspberry Pi and connect
Raspberry Pi to a display. Keep moving hand over the sensor to be able to
measure range. & Display prints out different readings recorded from moving
hand over the sensor \\
\bottomrule
\caption{Time of Flight Sensor Tests}
\label{table:tof-tests}
\end{longtable}

\subsubsection{Software Tests}

\paragraph{Message Format}

The unit tests described in table \ref{table:msg-frmt-tests} will be replicated
for each of the message classes (AccessRequestMessage, AccessResponseMessage,
InformationRequestMessage, InformationResponseMessage and
DoorStateUpdateMessage).

\begin{longtable}[htb]{>{\centering\arraybackslash}m{0.75cm}|>{\centering\arraybackslash}m{4cm}|>{\centering\arraybackslash}m{4.5cm}|>{\centering\arraybackslash}m{4cm}}
\toprule
Test & Description & Setup & Expected Result \\
\midrule
1 & Test the constructor & Create a new message object with the constructor &
Fields should be set to the values passed into the contructor \\
\hline
2 & Test convertion to byte array & Call the to\_bytes() method on a message
object & The resulting encoded mesage should be in the correct format \\
\hline
3 & Test parsing of valid packet & Use the \_parse() method to create a message
object from a bytes object & The fields of the message object should have been
parsed properly \\
\hline
4 & Test parsing of invalid packet & Attempt to use the \_parse() method to
create a message object from a bytes object with an incorrect length & A
MessageException should be thrown \\
\bottomrule
\caption{Message Format Class Tests}
\label{table:msg-frmt-tests}
\end{longtable}

\subsubsection{Database Tests}

The following features are to be tested in order to test the integrity of the
database:

\begin{itemize}
    \item Control server populating right tables 
    \item Proper formatted data uploaded to appropriate fields
    \item Records of the temperature readings of all employees are accessible
          for at least 14 days
\end{itemize}

\paragraph{Scenario 1}
Populating sample test data access\_entry table.

\noindent
Features to be tested:
\begin{itemize}
    \item Control server populating right tables
\end{itemize}

\noindent
Initial setup: set up Raspberry pi connected to monitor keyboard and mouse.
Open command prompt on Raspberry Pi.

\noindent
Test scenario steps:
\begin{enumerate}
    \item Invoke SQLite by typing this command in the command prompt
\begin{lstlisting}
sqlite3
\end{lstlisting}
    \item Open the database by typing this command 
\begin{listing}[H]
\begin{minted}[]{text}
.open project_database.db
\end{minted}
\end{listing}
    \item Populate data to employee access\_entry table
\begin{listing}[H]
\begin{minted}[]{sql}
INSERT INTO access_entry
    VALUES(19234,time('now'),date('now'), "S_entry",
           "Yes", 38.0, "Authorized");
INSERT INTO access_entry
    VALUES(19437,time('now','-12 minutes'),date('now'),
          "E_entry", "Yes", 38.3, "Authorized");
INSERT INTO access_entry
    VALUES(19632,time('now','-1 hour'),date('now'),
           "N_entry", "Yes", 39.2, "Unauthorized");
INSERT INTO access_entry
    VALUES(19957,time('now','-5 minutes'),date('now'),
           "W_entry", "Yes", 38.5, "Authorized");
\end{minted}
\end{listing}
    \item Query data using \lstinline{Select*} command
\begin{listing}[H]
\begin{minted}[]{sql}
SELECT * FROM access_entry;
\end{minted}
\end{listing}
\end{enumerate}

\noindent
Expected test scenario result: table populated with test sample data and query
outputs.

\begin{lstlisting}
19234 | 09:30:23 | 2020-10-26 | S_entry | Yes | 38.0 | Authorized
19437 | 09:18:23 | 2020-10-26 | E_entry | Yes | 38.3 | Authorized
19632 | 08:30:23 | 2020-10-26 | N_entry | Yes | 39.2 | Unauthorized
19957 | 09:25:23 | 2020-10-26 | W_entry | Yes | 38.5 | Authorized
\end{lstlisting}

\paragraph{Scenario 2}
Populating sample test data nfc\_and\_employer\_id table.

\noindent
Features to be tested:
\begin{itemize}
    \item Proper formatted data uploaded to appropriate fields
\end{itemize}

\noindent
Initial setup: set up Raspberry pi connected to monitor keyboard and mouse.
Open command prompt on Raspberry Pi.

\noindent
Test scenario steps:
\begin{enumerate}
    \item Invoke SQLite by typing this command in the command prompt
\begin{listing}[H]
\begin{minted}[]{bash}
sqlite3
\end{minted}
\end{listing}
    \item Open the database by typing this command
\begin{listing}[H]
\begin{minted}[]{sql}
.open project_database.db
\end{minted}
\end{listing}
    \item Populate data to employee access\_entry table
\begin{listing}[H]
\begin{minted}[]{sql}
INSERT INTO access_entry
    VALUES(19234,time('now'),date('now'), "S_entry",
           "Yes", 38.0, "Authorized");
INSERT INTO access_entry
    VALUES(19437,time('now','-12 minutes'),date('now'),
           "E_entry","Yes", 38.3, "Authorized");
INSERT INTO access_entry
    VALUES(19632,time('now','-1 hour'),date('now'),
           "N_entry","Yes", 39.2, "Unauthorized");
INSERT INTO access_entry
    VALUES(19957,time('now','-5 minutes'),date('now'),
           "W_entry","Yes", 38.5, "Authorized");
\end{minted}
\end{listing}
    \item Query data using \lstinline{Select*} command
\begin{listing}[H]
\begin{minted}[]{sql}
SELECT access_time, access_date FROM access_entry;
\end{minted}
\end{listing}
\end{enumerate}

\noindent
Expected test scenario result: table populated with test sample data and query
outputs.

\begin{listing}[H]
\begin{minted}[]{text}
09:30:23 | 2020-10-26
09:18:23 | 2020-10-26
08:30:23 | 2020-10-26
09:25:23 | 2020-10-26
\end{minted}
\end{listing}

\paragraph{Scenario 3}
Populating sample test data access\_entry table.

\noindent
Features to be tested:
\begin{itemize}
    \item Records of the temperature readings of all employees are accessible
          for at least 14 days
\end{itemize}

\noindent
Initial setup: set up Raspberry pi connected to monitor keyboard and mouse.
Open command prompt on Raspberry Pi.

\noindent
Test scenario steps:
\begin{enumerate}
    \item Invoke SQLite by typing this command in the command prompt
\begin{listing}[H]
\begin{minted}[]{bash}
sqlite3
\end{minted}
\end{listing}
    \item Open the database by typing this command
\begin{listing}[H]
\begin{minted}[]{sql}
.open project_database.db
\end{minted}
\end{listing}
    \item Populate data to employee access\_entry table
\begin{listing}[H]
\begin{minted}[]{sql}
INSERT INTO access_entry
    VALUES (19234, time('now'), date('now'), "S_entry",
            "Yes", 38.0, "Authorized");
INSERT INTO access_entry
    VALUES (19437, time('now','-12 minutes'),
            date('now', '-1 day'), "E_entry", "Yes", 38.3,
            "Authorized");
INSERT INTO access_entry
    VALUES (19632, time('now','-1 hour'),
            date('now', '-3 days'), "N_entry", "Yes", 39.2,
            "Unauthorized");
INSERT INTO access_entry
    VALUES (19957, time('now','-5 minutes'),
            date('now', '-7 days'), "W_entry", "Yes", 38.5,
            "Authorized");
INSERT INTO access_entry
    VALUES (19234, time('now'), date('now', '-13 days'),
            "S_entry", "Yes", 38.0, "Authorized");
INSERT INTO access_entry
    VALUES (19437, time('now','-12 minutes'),
            date('now', '-14 days'), "E_entry", "Yes", 38.3,
            "Authorized");
INSERT INTO access_entry
    VALUES (19632, time('now','-1 hour'),
            date('now', '-14 days'), "N_entry", "Yes", 39.2,
            "Unauthorized");
INSERT INTO access_entry
    VALUES (19957, time('now','-5 minutes'),
            date('now', '-15 days'), "W_entry", "Yes", 38.5,
            "Authorized");
\end{minted}
\end{listing}
    \item Query data using \lstinline{Select*} command
\begin{listing}[H]
\begin{minted}[]{sql}
SELECT * FROM access_entry WHERE access_date > date('now');
\end{minted}
\end{listing}
\end{enumerate}

\noindent
Expected test scenario result: table populated with test sample data and query
outputs.

\begin{listing}[H]
\begin{minted}[]{text}
19437 | 09:18:23 | 2020-10-25 | E_entry | Yes | 38.3 | Authorized
19632 | 08:30:23 | 2020-10-23 | N_entry | Yes | 39.2 | Unauthorized
19957 | 09:25:23 | 2020-10-19 | W_entry | Yes | 38.5 | Authorized
19234 | 09:30:23 | 2020-10-13 | S_entry | Yes | 38.0 | Authorized
19437 | 09:18:23 | 2020-10-12 | E_entry | Yes | 38.3 | Authorized
19632 | 08:30:23 | 2020-10-12 | N_entry | Yes | 39.2 | Unauthorized
19957 | 09:25:23 | 2020-10-11 | W_entry | Yes | 38.5 | Authorized
\end{minted}
\end{listing}


\subsection{Final Demo}

Table \ref{table:final-tests} lists the scenarios that we will test during the
final demo. These scenarios have been chosen to demonstrate the functional
requirements as listed in section \ref{sec:problem-statement}.

% Functional requirements:
% - Control access to a building using security badges.
% - Require users to present their security badge when they enter or exit the
%   building in order to track the number of users in the building.
% - Measure user's temperatures when they are entering the building in order to
%   determine if they have possible symptoms.
% - The door node should have a range sensor to determine whether users are in
%   an appropriate position for a temperature reading.
% - Do not allow more users to enter the building if the a preset maximum
%   capacity has been reached.
% - An multicoloured LED at each door node should indicate be used to indicate
%   the status of the door node. The LED should be normally red when the door is
%   locked and should change to green when the door is unlocked. The LED should
%   be orange when in the process of taking a temperature reading if the user is
%   not within an appropriate range of the temperature sensor.


\begin{longtable}[htb]{>{\centering\arraybackslash}m{3cm}|>{\centering\arraybackslash}m{3.5cm}|>{\centering\arraybackslash}m{3cm}|>{\centering\arraybackslash}m{3.5cm}}
\toprule
Description & Requirement & Procedure & Expected Result \\
\midrule
Ideal Building Entry & Control access to a building using security badges,
measure users' temperatures when they are entering the building, change LED 
colours & Present a valid security badge to the incoming reader, provide a
normal temperature reading the temperature sensor & The electronic door lock
should be actuated to allow entry, the access should be logged in the database,
the count of people in the building should incremented and the LED at the door
node should change to green temporarily \\
\hline
Ideal Building Exit & Require users to present their security badge when exiting
the building in order to track the number of users in the building & Present a
valid building security badge to the outgoing reader at a door node & The door
node should be actuated to allow the user to exit, the exit should be logged in
the database and the count of people in the building should be decremented \\
\hline
Attempted building entry with invalid security badge & Control access to a
building using security badges & Present an invalid security badge to the
incoming reader & The electronic door lock should not be actuated and entry
should not be allowed, the attempted access should be logged in the database and
the LED at the door node should remain red \\
\hline
Attempted building entry with fever & Measure user's temperature when they are
entering the building & Present a valid security badge to the reader but then
present a temperature that is too high & The electronic door lock should not be
actuated and entry should not be allowed, the attempted access should be logged
in the database and the LED at the door node should remain red \\
\hline
Attempted building entry with low temperature & Measure user's temperature when
they are entering the building & Present a valid security badge to the reader
but then present a temperature that is too low & The electronic door lock should
not be actuated and entry should not be allowed, the attempted access should be
logged in the database and the LED at the door node should remain red \\
\hline
Attempted building entry when building is at capacity & Do not allow users to
enter the building if a maximum capacity has been reached & Set the maximum
capacity of the system to one users, enter with a valid security badge and
temperature reading then attempt to enter with a second valid security badge &
The electronic door lock should not be actuated and entry should not be allowed,
the attempted access should be logged in the database and the LED at the door
node should remain red \\
\hline
Temperature measurement from incorrect distance & Determine whether users are in
an appropriate position for a temperature reading & Present a valid security
badge to the incoming reader but initially stand approximately one meter away
from the temperature sensor, slowly move closer until a temperature reading is
taken & The LED should be illuminated orange and the electronic door lock should
not be actuated until the user is within an appropriate range of the infrared
temperature sensor \\
\bottomrule
\caption{Final Demo Tests}
\label{table:final-tests}
\end{longtable}

