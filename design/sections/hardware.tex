The door node is the hardware that will be located at each door. It consists of
a NFC security badge reader, infrared temperature sensor, time of flight range
sensor and elextronic door lock, all connected to a Raspberry Pi.

Figure \ref{fig:door-node-schematic} shows a schematic of the door node.

\begin{figure}[!htb]
\centering
\includegraphics[width=\textwidth]{figures/door-node-schematic.png}
\caption{Door Node Schematic}
\label{fig:door-node-schematic}
\end{figure}

\subsection{Infrared Temperature Sensor}

In order to detect fever we need to measure the temperature of the users of our
system. We have determined that a forehead temperature measurement using an
infrared temperature sensor is the best way to obtain sufficiently accurate
contactless temperature measurements.

We plan to use the TPiS 1T 1256 L5.5 thermopile from Excelitas Technologies.
This temperature sensor was chosen after a comparison of available infrared
temperature sensors.

The TPiS 1T 1256 L5.5 offers medical grade accuracy within the temperature range
expected for human skin. It also provides accurate ambient temperature
measurments. The optics of the sensor provide a relativly narrow 5 degree field
of view which is appropriate for measuring the fairly small target of a human
forehead from a reasonable range.

The TPiS 1T 1256 L5.5 can be powered from the Raspberry Pi's 3.3 volt rail and
features a single pin digital interface of readout of the temperature data. This
digital interface requires more accruate timing than can easily be accomplished
with a Raspberry Pi so a small 8 bit microcontroller will be used with minimal
firmware that converts betweem the thermopile's digital interface and an I2C
interface.

Figture \ref{fig:ir-test-circuit} shows the test circuit for the infrared
temperature sensor.

\begin{figure}[!htb]
\centering
\includegraphics[width=0.75\textwidth]{figures/ir-test-circuit.png}
\caption{Test Circuit for Infrared Temperature Sensor}
\label{fig:ir-test-circuit}
\end{figure}

\subsection{NFC Security Badge Reader}

The T GPIO extension board used is coloured differently and indicates GPIO
rather than \# before the numbers. MFRC 522 RFID was selected due to it being
available in a kit that I already had access to and having provided code. 

\begin{figure}[!htb]
\centering
\includegraphics[width=\textwidth]{figures/nfc-test-circuit.png}
\caption{Test Circuit for NFC Security Badge Reader}
\label{fig:nfc-test-circuit}
\end{figure}

\subsection{Electronic Door Lock}

\subsection{Time of Flight Range Sensor}

\begin{figure}[!htb]
\centering
\includegraphics[width=0.6\textwidth]{figures/tof-test-circuit.png}
\caption{Test Circuit for Time of Flight Sensor}
\label{fig:tof-test-circuit}
\end{figure}


\subsection{Control Server}

The control server hardware will consist of a Raspberry Pi connected to a
monitor, keyboard and mouse.

